\documentclass{article}
\usepackage[a4paper, margin=3cm]{geometry}
\usepackage[brazilian]{babel}
\usepackage[utf8]{inputenc}

\title{IA para o \textit{Gomoku} utilizando poda $\alpha$-$\beta$}
\author{Matheus Silva P. Bittencourt \and Vinicius Macelai}
\date{\today}

\begin{document}

\maketitle

\section{Modelo dos estados}

Como é de se esperar, o jogo será modelado como um grafo para o uso do
algoritmo minmax com poda $\alpha$-$\beta$. Os vértices desse grafo são
possíveis tabuleiros do jogo, ou seja, uma matriz que representa onde as peças
colocadas previamente estão. As arestas vão ser as jogadas, isto é, o ato de
colocar uma peça no tabuleiro, que levará de um vértice (estado) a outro.

\section{Heurística}

Para a função de heurística, a ideia é identificar padrões no tabuleiro que
levam a vitória, por exemplo, 4 peças seguidas (na horizontal, vertical ou
diagonal) sem nenhuma peça do adversário nas laterais da sequência, leva a uma
vitória automática. Em \cite{allis1993go}, são definido alguns padrões
(classes) que indicam uma posição favorável ao jogador:

\begin{enumerate}
	\item \texttt{OXXXX\_}
	\item \texttt{\_XXXX\_}
	\item \texttt{\_XXX\_}
	\item \texttt{O\_XXX\_\_}
	\item \texttt{\_X\_XX\_}
\end{enumerate}

sendo que o ``\_'' representa uma casa vazia e um possível movimento de defesa
do adversário. Todas as situações acima, são situações em que o adversário
precisa fazer um movimento de defesa (se for a vez dele) ou ele perderá, e na
situação 2 o jogo já está ganho. Um tabuleiro que possua duas das situações
acima garante a vitória, visto que o adversário teria que defender as duas,
porém só tem uma jogada, no melhor caso.

Para refinar mais a função de heurística, julgamos que é necessário também
considerar situações prévias as acima, que são mais fáceis de alcançar, como a
situação \texttt{\_XX\_} (situação 6). Entretanto, uma situação como essa deve
valer consideravelmente menos que as acima, já que é preferível ter a situação
3, por exemplo, do que o tabuleiro cheio de situações 6.

Visto que o tabuleiro possui dimensão $15\times15$ ($225$ casas) e que no
máximo ocuparemos a metade dele (113 casas), isso possibilitará que sejam
formadas no máximo 56 duplas, ou 37 triplas. Dando valores numéricos as
situações, a situação 6 pode valer 1 ponto toda vez que for encontrada, assim,
as situações mencionadas acima devem valer no mínimo 56 pontos, já que uma
tripla, por exemplo, deve valer mais que o tabuleiro cheio de duplas. Já a
situação 2, deve valer $56 \times 37$, novamente, devido ao fato de que um
tabuleiro com uma quadrupla deve valer mais que um tabuleiro cheio de triplas.
Uma possível fórmula para a heurística, apenas considerando as peças do PC,
seria:

$$h(x) = s_2 \cdot 56 \cdot 37 + (s_1 + s_3 + s_4 + s_5) \cdot 56 + s_6$$

sendo que $s_n$ é o número de ocorrências da situação $n$. O valor também deve
ser computado para o adversário e subtraído. Talvez seja necessário considerar
o número de jogadas para o valor da heurística, sendo que com menos jogadas o
valor deve ser maior.

\section{Utilidade}

A função de utilidade deve levar em conta as situações de vitória, e usar a
função de heurística para refinar as diferentes situações de vitória, por
exemplo, uma vitória em que as chances eram maiores deve ter um valor de
utilidade maior, diferente de uma vitória ``apertada''. Além disso, qualquer
vitória deve ter um valor de utilidade maior que o maior valor possível para a
heurística. Podemos escrever a função de utilidade como:

$$u_v(x) = v + h(x)$$

sendo que $v$ seja suficientemente grande para que uma vitória possua sempre um
valor maior que qualquer outro tabuleiro. Numa situação de derrota, podemos
apenas inverter o valor de $m$: $u_d(x) = - v + h(x)$.

\section{Otimizações planejadas}

Para aproveitar melhor as podas, e fazer com que elas aconteçam mais
frequentemente, vamos sempre começar a buscar e construir o grafo de jogadas a
partir das casas vizinhas as peças já posicionadas pelo PC, visto que colocar
peças nessas posições vai aumentar a função de heurística, fazendo com que as
podas ocorram mais frequentemente, especialmente quando a busca alcançar as
posições menos favoráveis.

\bibliographystyle{plain}
\bibliography{relatorio-1}

\end{document}

\documentclass{article}
\usepackage[a4paper, margin=3cm]{geometry}
\usepackage[brazilian]{babel}
\usepackage[utf8]{inputenc}
\usepackage{indentfirst, hyperref}

\title{IA para o \textit{Gomoku} utilizando poda $\alpha$-$\beta$}
\author{Matheus Silva P. Bittencourt \and Vinicius Macelai}
\date{\today}

\begin{document}

\maketitle

\section{Decisões de Projeto}

Foi utilizado uma implementação do jogo Gomoku disponível em
\href{https://github.com/zambonin/multivac/tree/master/gomoku}{github.com/zambonin/multivac},
com algumas melhoria em sua interface. Para executar o jogo basta executar o
arquivo \texttt{main.py} com o comando \texttt{\$ python main.py}.

\begin{itemize}

	\item \textbf{Organização do projeto}
		\begin{itemize}

			\item \texttt{game/game.py} controla a parte de acesso aos modos de
			jogo e validade das entradas, bem como todo o fluxo de controle,
			comunicando-se com o tabuleiro e chamando métodos necessários para
			que jogadores humanos e artificial possam realizar suas ações no
			jogo.

			\item \texttt{game/board.py} realiza a implementação de um
			tabuleiro para um jogo de \textit{gomoku}, junto com seus métodos
			para a integração com o algoritmo de busca de decisão, usando
			recursos da linguagem escolhida, tais como \textit{list
			comprehensions e generators}. É nessa classe que é implementada a
			função de heurística/utilidade.

			\item \texttt{ab\_prunning.py} é responsável por construir árvores
			de jogadas recursivamente de forma dinâmica e implícita, dado o
			tabuleiro atual, e de acordo com a profundidade desejada, decidir
			qual será a melhor jogada possível respeitando a função de
			heurística/utilidade.

		\end{itemize}

	\item \textbf{Detecção de fim de jogo e sequências}
		A detecção de fim de jogo é feita a partir do método \texttt{victory},
		que percorre toda a matriz transformada em string do tabuleiro
		procurando por quíntuplas do mesmo jogador. Após cada jogada, essa
		verificação é aplicada. De modo similar, sequências são verificadas com
		expressões regulares.

	\item \textbf{Principais métodos}
		\begin{itemize}

			\item \texttt{ab\_pruning:} implementa o algoritmo minimax com
			poda.

			\item \texttt{evaluate:} calcula o valor de heurística ou utilidade
			da configuração atual do tabuleiro.

			\item \texttt{adjacents:} utiliza de um gerador para fornecer todos
			os estados do tabuleiro.

		\end{itemize}

\end{itemize}

\section{Limitações}

O software responde bem ao descer 2 níveis, com tempo de resposta em torno de
segundos, já aumentado para 3 níveis o tempo de resposta aumenta para cerca de
35 segundos. O gargalo do algoritmo apresenta-se na função \texttt{evaluate},
pois é necessário buscar todos os padrões com regex, que é uma operação
custosa.

Além disso, acabamos não implementando uma das otimizações planejadas, que era
começar a busca de jogadas, por jogadas ao redor das peças já existentes do PC.
Ainda assim implementamos uma maneira de otimizar as podas, realizando a busca
em espiral pelo tabuleiro, ou seja, primeiro o algoritmo analisa jogadas
válidas no centro do tabuleiro e depois nas bordas. Essa otimização pode ser
vista no método \texttt{adjacents} da classe \texttt{Board}.

\section{Implementação da função de heurística/utilidade}

As funções de heurística e utilidade, foram implementadas no mesmo método
(\texttt{Board.evaluate}) com o uso do módulo \texttt{re} da linguagem Python,
que implementa algumas utilidades para reconhecimento de \textit{strings}
utilizando expressões regulares. Optamos por essa solução, visto que, a nossa
função de heurística consistia basicamente em encontrar padrões no tabuleiro do
jogo, como descrito no primeiro relatório.

No arquivo \texttt{game/board.py} podemos encontrar as expressões regulares
utilizadas para encontrar uma situação de vitória no tabuleiro, e também, as
situações que contribuem para o valor da heurística. Utilizando estas
expressões, podemos contar quantas vezes cada situação ocorre no tabuleiro e
por sua vez calcular o valor de heurística.

Para isso, ocorre um pré-processamento da matriz, que representa o tabuleiro,
transformando ela em uma \textit{string} onde cada linha corresponde a uma
linha da matriz. Claro que não podemos analisar somente a matriz da maneira
como a vemos na interface gráfica, para isso, os métodos \texttt{\_columns,
\_diagonals, \_antidiagonals} foram implementados, que no caso retornam a
matriz do tabuleiro ``rotacionado'', possibilitando o reconhecimento das
situações, além das linhas, nas colunas e diagonais.

\end{document}
